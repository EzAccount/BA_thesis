\documentclass[a4paper]{article}
\usepackage[14pt]{extsizes}
\usepackage[utf8]{inputenc}
\usepackage[russian]{babel}
\usepackage{setspace,amsmath}
\usepackage[left=20mm, top=15mm, right=15mm, bottom=15mm, nohead, footskip=10mm]{geometry}
\usepackage{mathtools}
\usepackage{graphicx}
\usepackage{amssymb,amsfonts,amsmath,mathtext}
\usepackage{enumerate,float,indentfirst}
\usepackage{hyperref}
\usepackage[T2A]{fontenc}
\usepackage{tikz}	
\usepackage{ntheorem}
\usepackage{cleveref}
\usepackage{wrapfig}

\usepackage{listings}
\usepackage{chngcntr}
\usetikzlibrary{automata,arrows,positioning,calc}
 \DeclarePairedDelimiter\floor{\lfloor}{\rfloor}
\begin{document}
\section*{Слайд 1}
В данной квалификационной работе изучается модель Изинга со случайным вмороженным беспорядком, представленного случайной величиной внешнего магнитного поля. 
В таких системах можно говорить о вероятностных распредлениях физических величин и о усредненных по беспорядку средних значениях физических величин. В данной работе мы будет изучать среднюю энергию основного состояния. Отметим, что энергия -- самоусредняющаяся величина в данной модели, т.е. при увеличинии числа спинов дисперсия энергии стремится к нулю.

\section*{Слайды 2-3}
Отделяя минимизацию по последнему спину можно представить энергию основного состояния как функцию последнего спина. Удалось написать следующие рекуррентные соотношения, которые позволят нам посчитать энергию.

\section*{Слайд 4}
Рассмотрим конкретное распределение, представляющее собой простое трехточечное распределение. Введем параметр $k$.

\section*{Слайд 5}
Система рекуррентных соотношений для $H$ имеет инвариантное множество. Таким образом задача о поиске энергии основного состояния сводится к следующей марковской цепи, которая графически изображена на слайде. Переходы $p_2$ не отображены чтобы не засорять рисунок. Разделим условно эту цепь на две подцепи, состояния выделенные красным будем описывать вектором $a$, зеленым -- вектором $b$.
\section*{Слайд 6}
У данной марковской цепи есть предельное состояние. Для вероятностных векторов $a,b$ можно записать следующую систему уравнений.

\section*{Слайд 7}
Рассмотрим сначала более простой случай. При $k \in \mathbb{Z}$ цепь "схлопывается" в линейную цепочку отображенную на рисунке. Присвоим этим состояниям вектор $x$.

Система уравнений представляет собой разностное уравнение с граничными условиями. Разностное уравнение решается через соответствующее ему характеристическое уравнение. Не вдаваясь в детали, можно получить решение в следующем виде. Случай $p_1 = p_3$ -- тривиальный и ему соответствует равномерное распределение.

\section*{Слайды 8-9}
Вернемся к общему случаю. Можно сложить уравнения и получить уравнения для простой линейной цепочки.

\section*{Слайды 10-11}
Отметим так же, что можно сложить уравнения немного более хитрым образом и получить снова линейную цепочку, но размера $k+1$.
Таким образом мы получили выражения для переменных $a,b$ в терминах $x^{k+1}$ и $x^k$.

\section*{Слайд 12}
Энергия основного состояния с учетом данных векторов выглядит следующим образом. Заметим что она является кусочно-линейной функцией параметра $h$
\section*{Слайды 13-14-15}
Проведен аналогичные рассуждения для четырехточечного распределения и вырожденной цепочки
\section*{Слайд 16}
Получим выражение для энергии основного состояния, отметим что величина является кусочно-линейной функцией $h$. В ходе работы так же изучалось 5-точечное распределение, энергия и вероятности для которого получаются абсолютно аналогично.

\section*{Слайд 17}
Общеизвестным является тот факт. что при изучении собственных значений трансфер-матрицы модели Изинга, можно получить информацию о корреляционном радиусе моделируемой системы.
Возвращаясь к матрице перехода марковского процесса выдвинем следующую гипотезу.

В ходе выполнения работы были произведены попытки аналитической проверки этого факта, но формулы для вырожденного случая не соответствуют известным результатам. Было решено произвести численную симуляции. Численные эксперименты подтвердили гипотезу для невырожденного случая при вероятностях $p_1 = p_3 = 0.5$.

\section*{Слайд 18}
В литературе есть описание следующего интересного алгоритма. Сведем задачу о поиске минимальной энергии к минимальному разрезу на графе следующего вида. Здесь случайное поле представлено ребрами соединяющим физические спины (пронумерованы $i$) с фиктивными спинами $S-, S+$ описывающим направление этого поля. 
Далее проблему о минимальном $s-t$ разрезе сводится к проблеме максимизации потока. В ходе работы был реализован алгоритм Форда-Фалькенсона.
\section*{Слайд 19}
На данном слайде представлено сравнение реализации алгоритма Форда-Фалькенсона и алгоритма основанного на рекуррентных соотношениях. Выигрыш по времени очевиден, на большом количестве спинов алгоритм Форда-Фалькенсона использует слишком много памяти, в отличии рекуррсивного алгоритма, который не использует память.

\section*{Слайд 20}
Результаты. Спасибо за внимание.

\end{document}  