\section {Корреляционный  радиус системы}
Возвращаясь к матрице перехода, можно так же установить следующий факт:
	Пусть $r$ - корреляционный радиус системы. $\lambda_1 = 1 > \lambda_2 > \dots >\lambda_k$ - собственные значения матрицы перехода марковской цепи. Тогда:
	\begin{equation}
	r = -\frac{1}{\ln \lambda_2}
	\end{equation}
Чтобы убедиться в этом факте, рассмотрим довольно простое 2-точечное распределение следующего вида:
\begin{equation}
P(h_n) =
\begin{cases}
p_1 &\mbox{для } h_n= -h \\
p_1 &\mbox{для } h_n= h\\
\end{cases}
\end{equation}
Вычислим собственные значения матрицы перехода в соответствующем марковском процессе. Матрица имеет вид:
\begin{equation}
\begin{pmatrix}
0.5 & 0.5 & 0 & \dots & 0 \\
0.5 & 0 & 0.5 &\dots & 0\\
\vdots & \vdots & \vdots & \vdots & \vdots\\
0 & \dots & \dots & 0.5 & 0.5\\
\end{pmatrix}
\end{equation} 

А задача о спектре такой матрицы представляет собой разностное уравнение:
\begin{equation}
	0.5 x_{k-1} - \lambda x_{k} + 0.5 \lambda x_{k+1} =0,
\end{equation}
с граничными условиями:
\begin{align}
\begin{split}
	&x_0 = x_1\\
	&x_k = x_{k+1}\\
\end{split}
\end{align}

Общее решение имеет вид:
\begin{equation}
	x_i = A\nu_1^i + B\nu_2^i,
\end{equation}
где $\nu_1, nu_2$ - корни характеристического уравнения:
\begin{equation}
	\nu^2 - 2\lambda \nu + 1 = 0
\end{equation}

\begin{equation}
	\nu_{1,2} = \lambda + i \sqrt{1 - \lambda^2}
\end{equation}

Чтобы задача имела решение, система граничных условий должна быть вырожденной:

\begin{equation}
	\det \begin{pmatrix} 
	1 - \nu_1 & 1 - \nu_2 \\
	\nu_1^k - \nu_1^{k+1} &  \nu_2^k - \nu_2^{k+1} \\
	\end{pmatrix} = 0
\end{equation}

Или, вспоминая, что $\nu_1 \nu_2 = 1$
\begin{equation}
-\nu_1^{-k-1}-\nu_1 ^{1-k}+\nu_1 ^{k-1}+2 \nu_1 ^{-k}+\nu_1 ^{k+1}-2 \nu_1 ^k=0
\end{equation}

Получим, что:
\begin{equation}
\nu_1^k+1=0,
\end{equation}
т.е. $\nu$ это комплексные корни из $-1$. Вспоминая связь $\nu$ и $\lambda$ получаем, что $\lambda = \Re(\nu)$. Таким образом $\lambda_i = \cos(\frac{\pi + 2\pi i}{k}).$.

Этот результат [пока еще нет, но обязательно будет] согласуется с формулой из \cite{farhi1993correlation}
