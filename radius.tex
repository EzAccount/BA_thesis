\section {Корреляционный  радиус системы}

Общеизвестным является тот факт. что при изучении собственных значений трансфер-матрицы модели Изинга, можно получить информацию о корреляционном радиусе моделируемой системы. Более точно -- $Z_N =  Tr(T^N)$, $\frac{1}{r_c} = ln \frac{\lambda_max}{\lambda_{s.m}}$, где $\lambda_{max}, \lambda_{s.m}$ - первые два максимальных собственных значения трансфер-матрицы $T$. Подробности можно найти. например, в \cite{baxter}.

Возвращаясь к матрице перехода(матрице системы\ref{eq:matrix}), можно выдвинуть следующую гипотезу:
	Пусть $r$ - корреляционный радиус системы. $\lambda_1 = 1 > \lambda_2 > \dots >\lambda_k$ - собственные значения матрицы перехода марковской цепи. Тогда:
	\begin{equation}
	r = -\frac{1}{\ln \lambda_2}
	\end{equation}

Рассмотрим довольно простое 2-точечное распределение следующего вида:
\begin{equation}
P(h_n) =
\begin{cases}
0.5 &\mbox{для } h_n= -h \\
0.5 &\mbox{для } h_n= h\\
\end{cases}
\end{equation}
Вычислим собственные значения матрицы перехода в соответствующем марковском процессе. Рассмотрим сначала вырожденный случай. Матрица имеет вид (размер $k=\lceil2J/h \rceil$):
\begin{equation}
\begin{pmatrix}
0.5 & 0.5 & 0 & \dots & 0 \\
0.5 & 0 & 0.5 &\dots & 0\\
\vdots & \vdots & \vdots & \vdots & \vdots\\
0 & \dots & \dots & 0.5 & 0.5\\
\end{pmatrix}
\end{equation} 

А задача о спектре такой матрицы представляет собой разностное уравнение:
\begin{equation}
	0.5 x_{l-1} - \lambda x_{l} + 0.5 \lambda x_{l+1} =0,
\end{equation}
с граничными условиями:
\begin{align}
\begin{split}
	&x_0 = x_1\\
	&x_k = x_{k+1}\\
\end{split}
\end{align}

Общее решение имеет вид:
\begin{equation}
	x_i = A\nu_1^i + B\nu_2^i,
\end{equation}
где $\nu_1, nu_2$ - корни характеристического уравнения:
\begin{equation}
	\nu^2 - 2\lambda \nu + 1 = 0
\end{equation}

\begin{equation}
	\nu_{1,2} = \lambda + i \sqrt{1 - \lambda^2}
\end{equation}

Чтобы задача имела решение, система граничных условий должна быть вырожденной:

\begin{equation}
	\det \begin{pmatrix} 
	1 - \nu_1 & 1 - \nu_2 \\
	\nu_1^k - \nu_1^{k+1} &  \nu_2^k - \nu_2^{k+1} \\
	\end{pmatrix} = 0
\end{equation}

Упрощая:
\begin{equation}
(1-\nu_1)(\nu_2-1)(\nu_1^k-\nu_2^k)=0
\end{equation}
Или, вспоминая, что $\nu_1 \nu_2 = 1$ и учитывая, что $\nu_1, \nu_2 \neq 1$, $\nu_1 \neq 0$
\begin{equation}
\nu_1^{2k}-1 = 0
\end{equation}


т.е. $\nu$ это комплексные корни из $1$. Вспоминая связь $\nu$ и $\lambda$ получаем, что $\lambda = \Re(\nu)$. Таким образом $\lambda_i = \cos(\frac{\pi (i-1)}{k}).$.

Полученный аналитический результат не согласуется с формулой для вырожденного случая из \cite{farhi1993correlation}.
Удалось однако сделать численное моделирование для проверки гипотезы. Вычисления для невырожденного случая ($2J/h$ -- нецелое) при $p_1 = p_3= 0.5$ совпадают с вычислениями по формуле 3.4 из \cite{farhi1993correlation}. Листинг соотвествующей программы в приложении.