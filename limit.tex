
\section{Поиск предельного распределения}


Чтобы найти предельное распределение рассмотрим частный случай вышеописанной марковской цепи: $\frac{2J}{h} \in \mathbb{Z}$. В таком случае цепь вырождается, как показано на Рис. \ref{mark_chain}.
\begin{figure}[h!]
	\centering
	\begin{tikzpicture}[->, >=stealth', auto, semithick, node distance=2.5cm]
\tikzstyle{every state}=[fill=white,draw=black,thick,text=black,scale=1, minimum size=13mm]

\node[state]    (A)                     {$-J$};
\node[state]    (B)[right of=A]   {$-J+h$};
\node[draw=none]    (C)[right of=B]   {$\hspace{0.2in}\dots\hspace{0.2in}$};
\node[state]    (D)[right of=C]   {$J-h$};
\node[state]    (E)[right of=D]   {$J$};


\path
(A) edge[loop left]     node{$p_1+p_2$}         (A)
edge[bend left]  node{$p_3$}     (B)
(B) edge[bend left]               node{$p_3$}           (C)
edge[bend left]                node{$p_1$}           (A)
(C) edge[bend left]               node{$p_3$}           (D)
edge[bend left]                node{$p_1$}           (B)
(D)  edge[bend left]                node{$p_1$}           (C)
edge[bend left]                node{$p_3$}           (E)
(E)  edge[bend left]                node{$p_1$}           (D)
edge[loop right]     node{$p_2+p_3$}         	(E);

\end{tikzpicture}
\caption{Вырождение графа в частном случае}
\label{mark_chain}
\end{figure}
А уравнения для предельного распределения имеют вид:
\begin{align}
\begin{split}
x_i &= p_1 x_{i+1} + p_2 x_i + p_3 x_{i-1}\\
x_0 &= (p_1 + p_2) x_0 + p_1 x_1\\
x_k &= (p_2+p_3) x_k + p_3 x_{k-1},
\end{split}
\end{align}
где $x_k$ - собственный вектор матрицы перехода.

Система представляет собой разностное уравнение второго порядка с заданными граничными условиями. Будем искать решение в виде $x_n = A \lambda ^n$. Это приведёт нас к следующему характеристическому уравнению:
\begin{equation}
p_3 + (p_2 - 1) \lambda + p_1 \lambda^2 = 0
\end{equation}

Общее решение тогда запишется в виде:
   \begin{equation}
    x_n = A + B\left(\frac{p_3}{p_1} \right)^n=A+B t^n
    \end{equation}

Граничные условия накладывают ограничения на пространственно однородную часть ($A=0$) , и учитывая нормировку ($\sum x_i=1$), решение имеет вид ( при $t\neq 1$)
\begin{equation}
\label{sol}
x_n = \frac{1-t}{1-t^{k+1}} t^n
\end{equation}

Случай $t=1$ тривиальный, распределение состояний равномерное.
\begin{zam}
	Характеристическое уравнение имеет корень $1$ и для более сложных распределений. Этот факт позволяет построить точные решения для более широкого класса систем.
\end{zam}

Имея точное решение для некоторого частного случая, вернёмся к решению  общего случая.

Рассмотрим уравнения для предельного состояния:
\begin{align}
\begin{split}
a_i &= p_3 a_{i-1} + p_2 a_i + p_1 a_{i+1}\\
b_i &= p_3 b_{i+1} + p_2 b_i + p_1 b_{i-1}\\
a_0 &= (p_1 + p_2) a_0 + p_1 a_1 + p_1 b_k\\
b_0 &= (p_2 + p_3) b_0 + p_3 b_1 + p_3 a_k \\
a_k &= p_2 a_k + p_3 a_{k-1}\\
b_k &= p_2 b_k + p_1 b_{k-1}\\
\end{split}
\end{align}
и сложим уравнения для $a_i$ и $b_{k-i}$:
\vspace{1cm}

\begin{equation}
a_0+b_{k} = (p_1 + p_2) (a_0 +b_k)  + p_1 (a_1 + b_{k-1})
\end{equation}
\begin{equation}
a_i + b _{k-i} = p_3 (a_0 + b_{k-i+1}) + p_2 (a_i + b_{k-i}) + p_1 (a_2 + b_{k-i-1})
\end{equation}
\begin{equation}
a_k + b_0 = (p_2 + p_3) (a_k + b_0) +p_3(a_{k-1}+b_1)
\end{equation}
Схематично на рисунке это можно изобразить так:
\begin{figure}[h!]
\centering
\begin{tikzpicture}[->, >=stealth', auto, semithick, node distance=2.7cm,scale=1]
\tikzstyle{every state}=[fill=white,draw=black,thick,text=black,scale=1, minimum size=14mm]

\node[state]    (A)                     {$-J$};
\node[state ]    (B)[right of=A]   {$-J+h$};
\node[draw=none]    (C)[right of=B]   {$\hspace{0.1in}\dots\hspace{0.1in}$};
\node[state]    (D)[right of=C]   {$-J+kh$};
\node[state]    (E)[below of=A,yshift=-25pt]{$J-kh$};
\node[draw=none]    (F)[right of=E] {$\hspace{0.1in}\dots\hspace{0.1in}$};
\node[state ]    (G)[right of=F] {$J-h$};
\node[state ]    (H)[below of=D,yshift=-25pt]{$J$};

\path
(A) edge[loop left]     node{$p_1+p_2$}         (A)
edge[bend left]  node{$p_3$}     (B)
(B) edge[bend left]               node{$p_3$}           (C)
edge[bend left]                node{$p_1$}           (A)
(C) edge[bend left]               node{$p_3$}           (D)
edge[bend left]                node{$p_1$}           (B)
(D) edge               node{$p_3$}           (H)
edge[bend left]                node{$p_1$}           (C)

(H)  edge[bend left]                node{$p_1$}           (G)
edge[loop right]     node{$p_2+p_3$}         	(H)
(G) edge[bend left]                node{$p_1$}           (F)
edge[bend left] node{$p_3$} (H)
(F) edge[bend left] 		        node{$p_1$}         (E)
edge[bend left]node{$p_3$}         (G)

(E) edge 		        node{$p_1$}         (A)
edge[bend left]node{$p_3$}         (F);
\draw[red] ($(A)+(0,-2)$) ellipse (1.2cm and 3.5cm)node[yshift=4cm]{$a_0 + b_k$};
\draw[red] ($(D)+(0,-2)$) ellipse (1.4cm and 3.5cm)node[yshift=4cm]{$a_k + b_0$};
\end{tikzpicture}
\caption{"Склеивание" состояний марковской цепи}
\end{figure}
Таким образом для переменной $a_i + b_{k-i}$ получаем марковскую цепь, аналогичную рассмотренному частному случаю для параметра $k_{int}= k$.

Произведём теперь сложение уравнение другим образом. Перейдём к набору переменных $a_1+b_k, a_2+b_{k-1}, \dots a_k + b_1$, оставляя крайними состояниями $a_0, b_0$.  Схематично это выглядит следующим образом:
\begin{figure}[h!]
\centering
\begin{tikzpicture}[->, >=stealth', auto, semithick, node distance=2.7cm,scale=1]
\tikzstyle{every state}=[fill=white,draw=black,thick,text=black,scale=1, minimum size=14mm]

\node[state]    (A)                     {$-J$};
\node[state]    (B)[right of=A]   {$-J+h$};
\node[draw=none]    (C)[right of=B]   {$\hspace{0.1in}\dots\hspace{0.1in}$};
\node[state]    (D)[right of=C]   {$-J+kh$};
\node[state]    (E)[below of=A,yshift=-25pt]{$J-kh$};
\node[draw=none]    (F)[right of=E] {$\hspace{0.1in}\dots\hspace{0.1in}$};
\node[state]    (G)[right of=F] {$J-h$};
\node[state]    (H)[below of=D,yshift=-25pt]{$J$};

\path
(A) edge[loop left]     node{$p_1+p_2$}         (A)
edge[bend left]  node{$p_3$}     (B)
(B) edge[bend left]               node{$p_3$}           (C)
edge[bend left]                node{$p_1$}           (A)
(C) edge[bend left]               node{$p_3$}           (D)
edge[bend left]                node{$p_1$}           (B)
(D) edge               node{$p_3$}           (H)
edge[bend left]                node{$p_1$}           (C)

(H)  edge[bend left]                node{$p_1$}           (G)
edge[loop right]     node{$p_2+p_3$}         	(H)
(G) edge[bend left]                node{$p_1$}           (F)
edge[bend left] node{$p_3$} (H)
(F) edge[bend left] 		        node{$p_1$}         (E)
edge[bend left]node{$p_3$}         (G)

(E) edge 		        node{$p_1$}         (A)
edge[bend left]node{$p_3$}         (F);
\draw[rotate around={-37:(5,5)},red] ($(B)+(-0.1,-2.5)$) ellipse (1.5cm and 4cm)node[yshift=3.5cm, xshift=1cm]{$a_1 + b_k$};
\draw[rotate around={-37:(5,5)},red] ($(D)+(-0.1,-2.5)$) ellipse (1.5cm and 4cm)node[yshift=3.5cm, xshift=1cm]{$a_k + b_1$};
\end{tikzpicture}
\caption{Другой способ "склеивания" состояний}
\end{figure}
Для нового набора переменных $a_0, b_0, (a_i + b_{k-i+1})$, после сложения уравнений, получим марковскую цепь с параметром $k_{int} = k+1$.

Вспоминая решение ($\ref{sol}$) для этих цепей, получаем систему на необходимые вероятности:
\begin{align}
	\begin{split}
	\label{general}
	&a_0 = x^{k+1}_0\\
	&b_0 = x^{k+1}_{k+1}\\
	&a_0 + b_{k}= x^{k}_0\\
	&a_k + b_0 = x^k_k \\
	&a_1 + b_k = x^{k+1}_1\\
	&b_1 + a_k = x^{k+1}_1\\
	&a_{k-1} + b_1 = x^k_{k-1}\\
	&a_1+ b_{k-1} = x^k_{1}\\
	&\dots \\
	\end{split}
\end{align}
где верхний индекс $\cdot^k$ обозначает параметр $k_{int}$ цепи, для которой берётся предельное распределение. Такая система очевидно последовательно решается и позволяет выразить все вероятности.

