
\section{Поиск предельного распределения}


Чтобы найти предельное распределение рассмотрим частный случай вышеописанной марковской цепи: $\frac{2J}{h} \in \mathbb{Z}$. В таком случае инвариантное множество вырождается в цепочку следующего вида:

	\begin{tikzpicture}[->, >=stealth', auto, semithick, node distance=2.5cm]
\tikzstyle{every state}=[fill=white,draw=black,thick,text=black,scale=1, minimum size=13mm]

\node[state]    (A)                     {$-J$};
\node[state]    (B)[right of=A]   {$-J+h$};
\node[draw=none]    (C)[right of=B]   {$\hspace{0.2in}\dots\hspace{0.2in}$};
\node[state]    (D)[right of=C]   {$J-h$};
\node[state]    (E)[right of=D]   {$J$};


\path
(A) edge[loop left]     node{$p_1+p_2$}         (A)
edge[bend left]  node{$p_3$}     (B)
(B) edge[bend left]               node{$p_3$}           (C)
edge[bend left]                node{$p_1$}           (A)
(C) edge[bend left]               node{$p_3$}           (D)
edge[bend left]                node{$p_1$}           (B)
(D)  edge[bend left]                node{$p_1$}           (C)
edge[bend left]                node{$p_3$}           (E)
(E)  edge[bend left]                node{$p_1$}           (D)
edge[loop right]     node{$p_2+p_3$}         	(E);

\end{tikzpicture}

А уравнения для предельного распределения имеют вид:
\begin{align*}
x_i &= p_1 x_{i+1} + p_2 x_i + p_3 x_{i-1}\\
x_0 &= (p_1 + p_2) x_0 + p_1 x_1\\
x_k &= (p_2+p_3) x_k + p_3 x_{k-1},
\end{align*}
где $x_k$ - собственный вектор матрицы перехода.

Система представляет собой разностное уравнение второго порядка с заданными граничными условиями. Будем искать решение в виде $x_n = A \lambda ^n$. Это приведёт нас к следующему характеристическому уравнению:
\begin{equation*}
p_3 + (p_2 - 1) \lambda + p_1 \lambda^2 = 0
\end{equation*}

Общее решение тогда запишется в виде:
   \[ x_n = A + B\left(\frac{p_3}{p_1} \right)^n=A+B t^n\]

Граничные условия накладывают ограничения на пространственно однородную часть ($A=0$) , и учитывая нормировку ($\sum x_i=1$) решение имеет вид:
\begin{equation}
\label{sol}
x_n = \frac{1-t}{1-t^{k+1}} t^n
\end{equation}
\begin{zam}
	Характеристическое уравнение имеет корень $1$ и для более сложных распределений. Этот факт позволяет построить точные решения для более широкого класса систем.
\end{zam}

Имея точное решение для некоторого частного случая, вернёмся к решении в общем случае.
\newtheorem*{hyp}{Гипотеза}
\begin{hyp}
	Энергия является кусочно-непрерывной функцией $h$.
\end{hyp}

Чтобы убедиться в этом вернёмся к исходной марковской цепи, уравнения для предельного состояния:
\begin{align*}
a_i &= p_3 a_{i-1} + p_2 a_i + p_1 a_{i+1}\\
b_i &= p_3 b_{i+1} + p_2 b_i + p_1 b_{i-1}\\
a_0 &= (p_1 + p_2) a_0 + p_1 a_1 + p_1 b_k\\
b_0 &= (p_2 + p_3) b_0 + p_3 b_1 + p_3 a_k \\
a_k &= p_2 a_k + p_3 a_{k-1}\\
b_k &= p_2 b_k + p_1 b_{k-1}\\
\end{align*}
и сложим уравнения для $a_i$ и $b_{k-i}$:
\vspace{1cm}

		\[
a_0+b_{k} = (p_1 + p_2) (a_0 +b_k)  + p_1 (a_1 + b_{k-1})
\]
\[
a_i + b _{k-i} = p_3 (a_0 + b_{k-i+1}) + p_2 (a_i + b_{k-i}) + p_1 (a_2 + b_{k-i-1})
\]
\[
a_k + b_0 = (p_2 + p_3) (a_k + b_0) +p_3(a_{k-1}+b_1)
\]
Таким образом для переменной $a_i + b_{k-i}$ получаем марковскую цепь аналогичную рассмотренному частному случаю для параметра $k_{int}= k$.

Аналогично для набора переменных $a_0, b_0, (a_i + b_{k-i+1})$ получим марковскую цепь с параметром $k_{int} = k+1$.

Вспоминая решение $\ref{sol}$ для этих цепей получаем систему на необходимые вероятности:
	\begin{align*}
	&a_0 = x^{k+1}_0\\
	&b_0 = x^{k+1}_{k+1}\\
	&a_0 + b_{k}= x^{k}_0\\
	&a_k + b_0 = x^k_k,
	\end{align*}
где верхний индекс $\cdot^k$ обозначает параметр цепи, для которой берётся предельное распределение.

Для определения энергии состоянии нам достаточно знать только две крайние точки (для распределения содержащего $2n+1$ компонент - $2n$ крайних точек)  векторов $a, b$.

Все готово для получения компактного выражение энергии основного состояния:
		\[
		E = J + h p_3 x^{k+1}_{k+1}  + h p_1 x^{k+1}_0 + (J \text{mod} h) (p_3 (x^k_k - x^{k+1}_{k+1}) +p_1(x^k_0-x^{k+1}_0))
		\]

Таким образом мы показали верность гипотезы о непрерывности решения и получили явное выражение для энергии (в частных случаях согласующееся с известным точным решением \cite{nieuwenhuizen1986exactly})

Возвращаясь к матрице перехода, можно так же установить следующий факт:
\begin{ut}
	Пусть $r$ - корреляционный радиус системы. $\lambda_1 = 1 > \lambda_2 > \dots >\lambda_k$ - собственные значения матрицы перехода марковской цепи. Тогда:
	\[
	r = \lambda_2
	\]

\newpage


\end{ut}