\section{Введение}
	Модель Изинга (иногда модель Изинга-Ленца)  является одной из старейших и простейших нетривиальных моделей согласованного поведения, обладающая спонтанным нарушением симметрии. Она имеет большое количество различных приложений в различных областях науки  - от физики твёрдого тела до биологии. Гамильтониан модели в общем виде задаётся:
	\[
		H = - \sum_{ij} J_{ij}	S_i S_j - \sum_{i} H_i S_i,
	\]
	где $S_i = \pm 1$ - спиновые переменные. В более узком смыслом под моделью Изинга обычно понимают модель с трансляционно инвариантной константой связи, т.е. $J_{ij} = J(|r_i - r_j|)$.
	 Внешнее поле $H_i$ обычно считается однородным $H_i = H(r_i) = H$ или зависит гладко от $r_i$.
	 Даже при таких серьёзных ограничениях поведение модели сильно нетривиально. Кроме того,  реальные физические системы почти никогда не обладают трансляционной симметрией.
	 Наличие различных примеси и нарушения в решётках привели к модификации Гамильтониана: $J(|r_i-r_j) \to J(|r_i-r_j|) + \delta J_{ij}$ и $H \to H + h_i$. Под $\delta J_{ij}$ и $h_i$  мы понимаем случайные величины, с определенным заданным распределением. Обычно $h_i$ и $\delta J_{ij}$ имеют нулевое среднее и распределены независимо в каждой ячейке.

	Как было сказано раньше, модель Изинга со случайным полем имеет несколько интересных реализаций в природе. Самой хорошо экспериментально изученной (например \cite{fishman}) системой является система разбавленных антиферромагнетиков  в однородном внешнем поле, где комбинация примеси и внешнего поля приводит к эффекту случайного поля для ступенчатой намагниченности.

	Другими примерами реализации данной модели являются бинарные растворы в пористой среде \cite{de1984liquid}, системы с эффектом Яна-Тейлора \cite{graham1987random} и разбавленные фрустированные антиферрмагнетики \cite{fernandez1988random}. Хорошим обзорной работой на состояние исследований по этой теме есть например в \cite{nattermann1998theory}.
