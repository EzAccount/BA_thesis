\section{Введение}
	Модель Изинга (иногда модель Изинга-Ленца)  является одной из старейших и простейших нетривиальных моделей согласованного поведения, обладающих спонтанным нарушением симметрии. Она имеет богатые приложения в различных областях науки –– от физики твёрдого тела до биологии. Гамильтониан модели в общем виде задаётся следующим выражением:
	\begin{equation}
		H = - \sum_{ij}^N J_{ij}	S_i S_j - \sum_{i} H_i S_i,
	\end{equation}
	где $S_i = \pm 1$ –– спиновые переменные, $J_{ij}$ –– энергия взаимодействия спинов, $H_i$ –– внешнее магнитное поле. В данной модели мы считаем $J,H$ –– внешними параметрами системы. В более узком смысле под моделью Изинга обычно понимают модель с трансляционно-инвариантной константой связи, т.е. $J_{ij} = J(|r_i - r_j|)$.
	 Внешнее поле $H_i$ обычно считается однородным $H_i = H(r_i) = H$ или зависит гладко от $r_i$.
	 Даже при таких серьёзных ограничениях поведение модели сильно нетривиально. Кроме того,  реальные физические системы почти никогда не обладают трансляционной симметрией.
	 Наличие различных примесей и нарушения в решётках привели к модификации Гамильтониана: $J(|r_i-r_j) \to J(|r_i-r_j|) + \delta J_{ij}$ и $H \to H + h_i$. Под $\delta J_{ij}$ и $h_i$  мы понимаем случайные величины, с определенным заданным распределением. Обычно $h_i$ и $\delta J_{ij}$ имеют нулевое среднее и распределены независимо.

	Как было сказано выше, модель Изинга со случайным полем имеет несколько интересных реализаций в природе. Самой хорошо экспериментально изученной (например \cite{fishman}) системой является система разбавленных антиферромагнетиков  в однородном внешнем поле, где комбинация примеси и внешнего поля приводит к эффекту случайного поля для намагниченности.

	Другими примерами реализации данной модели являются бинарные растворы в пористой среде \cite{de1984liquid}, системы с эффектом Яна-Тейлора \cite{graham1987random} и разбавленные фрустированные антиферромагнетики \cite{fernandez1988random}. Хороший обзор работ и текущего состояния темы есть, например, в \cite{nattermann1998theory}.
	
	Мы рассмотрим отдельный класс систем со случайным полем, ограничившись взаимодействием только ближайших соседей, дискретным распределением $h_i$, нулевым постоянным полем и детерминированной пространственно однородной константной связи $J_i$. Гамильтониан такой системы имеет вид:
	\begin{equation}
		H = - J \sum_{n=2}^{N} S_{n-1} S_n - \sum_{n=1}^N h_n S_n
	\end{equation}
    В системе с замороженным беспорядком можно изучать как распределения случайных величин, так и усредненные значения по всем реализациям.Мы будем изучать величины вида $[X]$, где $[\cdot]$ -- усреднение по реализациям, а $X$ - некоторая физическая наблюдаемая. Будем называть величину самоусредняющейся, если относительная дисперсия $R_X = \frac{ [X^2]-[X]^2}{[X]^2} \to 0$, при  $ N\to \infty$. Вне критической точки, это свойство обуславливается центральной предельной теоремой ($R_X \simeq \frac{1}{N}$).
	Отметим, что энергия основного состояния (и некоторые другие величины) в изучаемой модели являются самоусредняющимися величинами.
	
	Интерес к такой модели возник у исследователей давно, и вызван он, в первую очередь, комбинацией эффектов случайности и фрустрации, которые отвечают большому количеству вырожденных или почти вырожденных основных состояний, аналогично моделям спинового стекла. Самой первой работой на эту тему можно считать статью \cite{fan1969one}. Дальнейшее развитие этой области связано с работой \cite{vannimenus1977theory}, обсуждающей явление фрустрации и проявления этого явления в моделях беспорядком. Первым результатом, имеющим непосредственное отношение к нашей работе, стоит считать работу
	\cite{derrida1978simple}, где приводится точное решение системы со случайным взаимодействием, полученное рекурсивным вычислением трансфер-матрицы в пределе нулевой температуры. Последующая работа \cite{PhysRevB.27.4503} посвящена решению системы со случайным полем, и является, насколько мне известно, первым решением для данного типа систем с определенными распределениями случайно величины $h$. Стоит так же отметить работу \cite{derrida1983singular}, посвященную режиму $J>h$. Первой работой в которой изучается поведение в режиме ненулевой температуры можно считать \cite{nieuwenhuizen1986exactly}, где изучается конкретный класс распределений $h$ (экспоненциальный). Позднее для нашей модели были так же изучены корреляционные функции (\cite{farhi1993correlation}, \cite{igloi1994correlations}). Самой поздней работой, которая мне известна, является статья \cite{hamasaki2004exact}, которая изучает случай $P(h=0) \neq 0$ (предыдущие работы были сфокусированы на случае случайного поля $\pm h$), однако использованный метод, основанный на статьях \cite{dress1995zero}, \cite{kadowaki1996exact}, так же опирается на написании рекуррентных соотношений в пределе нулевой температуры. Метод, использованный в данной работе, не использует предел нулевой температуры, что выгодно выделяет его на фоне предыдущих исследований.
	
