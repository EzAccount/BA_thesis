\section{Различные распределения}
Обсудим теперь на какие классы распределений можно обобщить наше решение. Ограничимся $k$-точечными дискретными распределениями.
Будем рассматривать только частный случай - вырожденную цепочку. Вышеописанная система уравнений (\ref{general}) позволяет построить решение в общем случае.

4-точечное асимметричное распределение:
\begin{equation*}
P(h_n) =
\begin{cases}
p_1 &\mbox{для } h_n= +2h \\
p_2 &\mbox{для } h_n= +h \\
p_3 &\mbox{для } h_n= 0\\
p_4 &\mbox{для } h_n = -h\\
\end{cases}
\end{equation*}

Для такого распределения получим разностное уравнение следующего вида:
\begin{equation}
	p_4 x_{i+1} + p_3 x_{i} + p_2 x_{i-1} +p_1 x_{i-2} = x_{i}
\end{equation}

Соответствующее характеристическое уравнение запишется как:
\begin{equation}
	p_1+ p_2\lambda + (p_3-1) \lambda^2 + p_4 \lambda^3 = 0
\end{equation}

Нетрудно заметить, что $\lambda=1$ является корнем этого уравнения. Тогда:
\begin{equation}
(\lambda - 1)(p_4 \lambda^2 - (p_2 + p_1)\lambda - p_1	) =0
\end{equation}

Общее решение можно записать в виде:
\begin{equation}
x_i = A \lambda_1^i + B \lambda_2^i + C,
\end{equation}
где $\lambda_{1,2}$ корни соответствующего квадратного уравнения.
Из граничных условий получим $A=-B$, $C=0$. Добавляя условие нормировки:
\begin{equation}
A \frac{1-\lambda_1^k}{1-\lambda_1} + B \frac{1-\lambda_2^k}{1-\lambda_2} =1
\end{equation}
получим:
\begin{equation*}
A=-B = \frac{(\lambda_1-1)(\lambda_2-1)}{\lambda_1(\lambda_1^k-1)(\lambda_2-1) - \lambda_2(\lambda_2^k-1)(\lambda_1-1)}
\end{equation*}

Рассмотрим теперь 5-модальное симметричное распределение:
\begin{equation*}
P(h_n) =
\begin{cases}
	p_1 &\mbox{для } h_n= +2h \\
	p_2 &\mbox{для } h_n= +h \\
	p_3 &\mbox{для } h_n= 0\\
	p_2 &\mbox{для } h_n = -h\\
	p_1&\mbox{для } h_n = -2h\\
\end{cases}
\end{equation*}
Разностное уравнение имеет вид:
\begin{equation*}
p_1 x_{i-2} + p_2 x_{i-1} + p_3 x_i + p_2 x_{i+1} + p_1 x_{i+2} = x_i
\end{equation*}
А соответствующее характеристическое уравнение:
\begin{equation*}
p_1 + p_2 \lambda + (p_3-1) \lambda^2 + p_2 \lambda^3 + p1 \lambda^4 = 0
\end{equation*}
Заметим, что $\lambda_1=1$ является корнем этого уравнения с кратностью 2. Тогда уравнение можно переписать в следующем виде:
\begin{equation*}
(p_1\lambda^2 - (2p_1+p_2) \lambda + p_1) (\lambda - 1)^2 =0
\end{equation*}
Общее решение разностного уравнения:
\begin{equation*}
x_i = A + B i + C \lambda_1^i + D \lambda_2^i
\end{equation*}
Учитывая граничные условия:
\begin{align*}
B&=0 \\
C&=A \frac{\left(\lambda_2^N-1\right)}{\lambda_1^N-\lambda_2^N} \\
D&=A\frac{\left(\lambda_1^N-1\right)}{\lambda_1^N-\lambda_2^N} \\
\end{align*}
И  условие нормировки:
\begin{equation*}
	A^{-1} = k+\frac{\left(\lambda_2^N-1\right)}{\lambda_1^N-\lambda_2^N}\lambda_1 \frac{\lambda_1^N-1}{\lambda_1-1}+\frac{\left(\lambda_1^N-1\right)}{\lambda_1^N-\lambda_2^N}\lambda_2 \frac{\lambda_2^N-1}{\lambda_2-1}
\end{equation*}

 

