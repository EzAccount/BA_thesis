\section{Численное моделирование}
Полученные формулы можно так же применить для численного моделирования.
Интересный способ численного подсчёта энергии основного состояния предложен в \cite{hartmann2004new}. Изложим здесь кратко основные идеи этого метода.

Сведём исходную задачу на решётке к задаче на графе следующего вида:
\begin{tikzpicture}[->, >=stealth', auto, semithick, node distance=2.7cm,scale=1]
\tikzstyle{every state}=[fill=white,draw=black,thick,text=black,scale=1, minimum size=14mm]

\node[state]    (A)                     {$S+$};
\node[state]    (B)[below of=A]   {$i$};
\node[state]    (C)[below of= B] {$S-$};
\node[state]    (D)[left of=B] {$i-1$};
\node[state]    (E)[left of=D]{$i-2$};
\node[draw=none]    (F)[left of=E]   {$\hspace{0.1in}\dots\hspace{0.1in}$};
\node[state]    (G)[right of=B] {$i+1$};
\node[state]    (H)[right of=G]{$i+2$};
\node[draw=none]    (I)[right of=H]   {$\hspace{0.1in}\dots\hspace{0.1in}$};
\draw[-] (F) to node {$J$} (E);
\draw[-] (E) to node{$J$} (D);
\draw[-] (D) to node{$J$} (B);
\draw[-] (B) to node{$J$} (G);
\draw[-] (G) to node{$J$} (H);
\draw[-] (H) to node{$J$} (I);

\draw[-] (A) to node{$h_{i-2}$} (E);
\draw[-] (A) to node{$h_{i}$} (B);
\draw[-] (A) to node{$h_{i+1}$} (G);
\draw[-](C) to node{$h_{i-1}$} (D);
\draw[-](C) to node{$h_{i+2}$} (H);
\end{tikzpicture}

Вдоль горизонтальной линии расположены физические спины, а $S+$ и $S-$ -- фиктивные спины, отвечающие за знак внешнего поля $h$. Узел, в котором поле положительно связывается ребром графа весом $h_i$ с фиктивным фиксированным спином $S+$(направленным вверх), а узел с отрицательным полем - соответственно, с $S-$(направленным вниз). Такой граф устанавливает соответствие между моделью Изинга со случайным полем и моделью Изинга со случайным взаимодействием и двумя фиксированными спинами. Исходная задача о поиске энергии и конфигурации основного состояния представляет задачу о минимальном $s-t$ разрезе. Согласно Теореме Форда — Фалкерсона (\cite{ford1962d}, \cite{ford1956maximal}, так же известная как min cut - max flow теорема) величина максимального потока равна пропускной способности минимального разреза, потому достаточно найти максимальный поток.

Воспользуемся стандартным известным алгоритмом для решения этой проблемы - алгоритмом проталкивания предпотока(push-relabel algorithm). Суть алгоритма описана например в книге Кормена (\cite{cormen2009introduction}). Алгоритм имеет сильно локальный характер, в отличие например от алгоритма Форда-Фалкерсона (\cite{ford1956maximal}) и использует $O(V^2 E)$ времени и $O(V^2)$ памяти. Выбор алгоритма отчасти связан с желанием обобщить успешные результаты рекуррентной формулы на двумерный случай (рекуррентные формулы, очевидно, носят сугубо локальный характер).

В ходе работы была разработана простейшая реализация этого алгоритма для экспериментальной оценки времени и вычислительных возможностей алгоритма. Для сравнения предлагается простейший алгоритм основанный на последовательном применении рекуррентной формулы (\ref{min_energy}). Результаты сравнения приведены в таблицах (\ref{table1}-\ref{table3}):

\begin{table}[h!]
\centering
\begin{tabular}{ |c|c|c| } 
	\hline
	Количество спинов & Push - Relabel & Рекур. соотношения \\
	\hline
	100 & 0.061 с & 0.002 c\\
	\hline
	1000 & 44.91 с & 0.010 c \\
	\hline
	2000 & 370.9 с & 0.01	5 c \\
	\hline
	3000 & 1216 с & 0.027 с \\
	\hline
\end{tabular}
	\caption{Время выполнения для малых полей ($|h|<J/2$)}
	\label{table1}
\end{table}
\begin{table}[h!]
	\centering
	\begin{tabular}{ |c|c|c| } 
		\hline
		Количество спинов & Push - Relabel & Рекур. соотношения \\
		\hline
		1000 & 0.086с & 0.01 c\\
		\hline
		100000 & 5.055 с & 0.07 c \\
		\hline
		1000000 & memory limit & 0.65 c \\
		\hline
	\end{tabular}
	\caption{Время выполнения для больших положительных полей ($h>J$,  тривиальный случай)}
	\label{table2}
\end{table}

\begin{table}[h!]
	\centering
	\begin{tabular}{ |c|c|c| } 
		\hline
		Количество спинов & Push - Relabel & Рекур. соотношения \\
		\hline
		1000 & 31.62 с & 0.0076 c\\
		\hline
		2000 & 258 с & 0.0142 c \\
		\hline
		3000 & 370.9 с & 0.015 c \\
		\hline
	\end{tabular}
	\caption{Время выполнения в общем случае}
	\label{table3}
\end{table}


	