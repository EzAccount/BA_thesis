\section{Выводы}
В ходе работы было получено точное решения для энергии основного состояния в системах с замороженным беспорядком для различных функций распределений внешнего поля.

Было установлено, что энергия является кусочно-линейной функцией дискретного шага поля. Также отметим, что связь корреляционного радиуса системы и спектра матрицы перехода выполняется для более широкого класса задач. Этот факт был установлен в ходе численных экспериментов с различными распределениями.

Был так же придуман новый алгоритм для вычисления энергии заданной конфигурации. Известный алгоритм работает за разумное время на размерах порядка $10^3$. При увеличении размеров и затрачиваемое время, и необходимая память становятся заметно большими. Алгоритм, построенный на рекуррентной формуле работает соответственно за $O(N)$ по времени и за $O(1)$ по памяти, так как необходимо только значение поля в текущей ячейке (нет необходимости хранить все поля).  

Важным направлением дальнейшего развития этой темы, безусловно, можно считать переход к двумерной системе. Здесь стоит сразу две задачи. Во-первых это получение точного решения для двумерной модели. На данный момент нет широко известных работ, описывающих (точно) энергию основного состояния для двумерной системы с замороженным беспорядком. 
Вторая задача - алгоритмическая. Вычисление энергии основного состояния двумерных систем с заданным беспорядком - важная с точки зрения приложений задача. Обобщение алгоритма, использующего рекуррентные формулы, должно работать быстрее, чем существующие аналоги.
